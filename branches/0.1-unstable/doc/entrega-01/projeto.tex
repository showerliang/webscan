\documentclass[a4paper, 12pt]{article}

\usepackage[brazilian]{babel}
\usepackage{ae,aecompl}
\usepackage[utf8]{inputenc}

\usepackage{fullpage}  
\usepackage{indentfirst}      % Indenta o primeiro paragrafo das sections
\usepackage{graphicx}         % Utilizado para inserir imagens
\usepackage{hyperref}         % Cria link entre o sumario e as sections
\usepackage{fancyhdr}         % Permite alterar o header e footer
\usepackage{mdwlist}          % Permite criar listas com menos espaçamento
\usepackage[includefoot]{geometry}

% Seta o estilo padrao
\pagestyle{fancy}

% Configura links
\hypersetup{colorlinks=true,linkcolor=black,citecolor=black}

% Seta espacamento
\addtolength{\parskip}{2mm}
\addtolength{\footskip}{15mm}

% Coloca no pé a pagina atual
\begin{document}

% Inclua as variáveis
\newcommand{\titulo}{Documento de Projeto}
\newcommand{\subtitulo}{Produto: DIGI-DOC}
\newcommand{\autor}{Sérgio Oliveira Campos}
\newcommand{\data}{\today}
\newcommand{\cidade}{Brasília}

% Numero do contrato caso seja consultor
\newcommand{\ncontrato}{1234567890-32}


% Inclua header
\lhead {
	\setlength{\unitlength}{5mm} 
	\begin{picture}(0,0) 
		\includegraphics[scale=0.75]{img/header.pdf} 
	\end{picture} 
	\textsf{\vspace{1cm}}
}

\chead{}
\rhead{}
\renewcommand{\headrulewidth}{0pt}


% Inluia a capa
\newpage
% $Id$
% ---------------------------------------------------------------------------
%
%  This is part of the relatorio-modelo.
%  Copyright (C) 2008 Interlegis
%  See the file relatorio.tex for copying conditions.
%

\textsf{\vspace{6cm}}
\begin{center}
  \noindent
  \huge{
    \textbf{\titulo}
  } \\
  \Large{
    \textbf{\subtitulo}
  }\\
  \large{
    \textbf{\subsubtitulo}
  }

  \vspace{9cm}

  \large{
    \textbf{\autor}\\
    \textsf{Contrato N$^{\circ}$: \numcontrato}\\
  }
\end{center}
\cfoot{}                        % retira o rodape

%
% Local variables:
%   mode: flyspell
%   TeX-master: "relatorio.tex"
% End:
%


% Seta o footer das paginas iniciais para numeros Romanos
\cfoot{\thepage}
\setcounter{page}{1}
\pagenumbering{Roman}

% Sum√°rio
\newpage
\tableofcontents

% Lista de figuras
\newpage
\listoffigures

% Lista de tabelas
%\newpage
%\listoftables

% Seta o footer das paginas para numeros convencionais
\clearpage
\setcounter{page}{1}
\pagenumbering{arabic}

% PARA INSERIR NOVAS PAGINAS BASTA ADICIONAR AQUI:
% Conteúdo
\section{Introdução}
\label{sec:intro}

Durante a primeira fase da elaboração do projeto {\it webscan}, foram realizadas as seguintes atividades:

\begin{description}
    \item[Casos de uso: ] Nessa atividade, descrita na seção \ref{sec:casos_de_uso}, foram elaborados diagramas de casos de uso, representando as principais interações entre atores e o sistema. Também foram elaborados os casos de uso completo-abstrato, de forma a dar detalhamento a cada caso de uso contido no diagrama;
    \item[Modelagem de dados: ] A modelagem de dados do projeto foi realizada para dar uma visão geral dos dados que serão tratados e como eles se relacionam. A seção \ref{sec:modelo_de_dados} apresenta o resultado obtido desse trabalho realizado. 
    \item[Interface: ] Para essa atividade, descrita com mais detalhes na seção \ref{sec:mockups}, foram elaboradas as candidatas às telas de interface do sistema ({\it mockups}) e também o curso de ações que um usuário pode realizar ({\it storyboards});
    \item[Elaboração dos {\it Web Services}: ] Essa atividade consistiu na
        espeficação dos métodos que compoem a interface de comunicação da
        aplicação com outros sistemas. Toda a especificação está disponível
        na seção \ref{sec:web_services}. 
    \item[Pesquisa de bibliotecas: ] Essa atividade (apresentada nas seções \ref{sec:pesquisa_libs} e \ref{sec:pesquisa_ocr}) consistiu na realização de uma pesquisa sobre as bibliotecas de digitalização e de reconhecimento de textos respectivamente.

\end{description}

\subsection{Terminologia}

\subsubsection{Atividade de Desenvolvimento}
Atividade de desenvolvimento se refere à quantidade de escritas (ou seja, código sendo atualizado/adicionado) em um sistema de controle de versões, quando disponível.

\begin{description}
    \item[Alta:] diversas atividades no último mês.
    \item[Baixa:] algumas atividades ao longo dos últimos 3 meses.
    \item[Parado:] não houve nenhuma atividade de escrita nos últimos 6 meses.
\end{description}

\subsubsection{OCR - Optical Character Recognition}

O OCR (Optical Character Recognition), ou Reconhecimento Óptico de Caracteres, 
é a tecnologia responsável pela obtenção de texto a partir de uma imagem. Durante
este projeto a tecnologia será empregada para gerar documentos indexáveis\footnote{Documentos indexáveis: Que
podem ser encontrados pelo sistema de busca}.

\subsubsection{JSON}
Algumas vezes é necessário que uma pequena informação seja transmitida entre aplicações, e o formato XML acaba
burocratizando demasiadamente este processo. Outro cenário é o de múltiplas requisições em um curto espaço de
tempo, que leva o cliente e o servidor a uma sobrecarga para executar o \emph{parser}, além de um uso de
excessivo da banda para a transmissão dos dados.

A padronização de um formato Javascript para a transfêrencia de dados poderia ser uma alternativa para
solucionar estes problemas, e foi por isso que no ano de 2002, \emph{Douglas Crockford}, engenheiro da
\emph{Yahoo! Inc.} propôs o formato JSON.

O principal objetivo era criar um padrão para troca de dados utilizando código Javascript, ou seja, em forma
textual, gerando o mínimo de texto possível, o que tornaria o formato leve e ao mesmo tempo fácil de ser
interpretado pelo navegador. Para isso algumas assertivas foram seguidas:

\begin{itemize}
    \item Não poderia ser uma linguagem de marcação;
    \item Não seria um formato de documento;
    \item Não permitiria a representação de funções;
    \item Não permitiria a representação de estruturas cíclicas.
\end{itemize}

No ano de 2006 o formato foi oficializado pelo \emph{Network Working Group} e apresentado oficialmente à
comunidade durante a conferencia \textbf{XML 2006}.

O padrão apresentado é basicamente composto de um objeto (Figura \ref{fig:json_obj}) que possui uma \emph{string}
descritiva e o seu valor, que pode assumir os formatos:

\begin{itemize}
    \item \emph{String}
    \item Número
    \item Vetor
    \item Objeto
    \item true, false e null
\end{itemize}

\begin{figure}[ht]
\begin{center}
\scalebox{0.6} {
    \includegraphics{img/json_obj.png}}
\end{center}
  \caption{Objeto JSON}
  \label{fig:json_obj}
\end{figure}

As definições detalhadas de cada um dos tipos e exemplos de código podem ser encontrados no site
http://www.json.org/.

\subsubsection{Web Services}

A W3C \footnote{http://www.w3c.org/} define \emph{web services} como um padrão que provê a
interoperabilidade entre duas aplicações de software, rodando sob diferentes plataformas e/ou
\emph{frameworks}.
A interoperabilidade fornecida pelos \emph{web services} é disponibilizada por meio de funções
ou mesmo objetos na web, de forma que estes possam ser chamados através de um \emph{HTTP
request} e sua resposta retornada através de um \emph{HTTP response}.
Para que uma aplicação consiga se comunicar com a outra, é necessário que ela conheça
e entenda o formato de entrada e saída de dados; para isso, é de costume que seja utilizado XML ou JSON.
Outro problema é que a aplicação deve saber qual o tipo de dados de um determinado
valor que chega a ela, e como ela implementa este valor. Este problema pode ser resolvido
de formas distintas; uma delas é a especificação trazer as informações necessárias; a outra
é o uso de um arquivo que traz esse tipo de informação, de tal forma que a aplicação
apenas leia este arquivo e faça as conversões necessárias. 



\section{Casos de Uso}
\label{sec:casos_de_uso}

Para o projeto, foram elaborados casos de uso do sistema. Na figura \ref{fig:casos_de_uso}
tem-se o diagrama de casos de uso. Na seção \ref{sec:casos_completos} tem-se os casos de uso
completo-abstrato, que indicam as principais atividades que acontecem em cada caso de uso, presentes no diagrama da figura \ref{fig:casos_de_uso}.

\subsection{Diagrama de Casos de Uso}
\begin{figure}[ht]
 \centering
  \includegraphics[scale=0.7]{img/use-case-diagram.pdf}
  \caption {Diagramas de casos de uso}
  \label{fig:casos_de_uso}
\end{figure}

\subsection{Casos de Uso Completos}
\label{sec:casos_completos}

\subsubsection{Caso de Uso: Digitaliza Documento}

\paragraph{Descrição:}
Nesse caso de uso o ator tem como função colocar um documento no {\it scanner} para digitalizá-lo.

\paragraph{Pré-condições:}
\begin{enumerate}
    \item Há um documento no {\it scanner};
    \item O {\it scanner} está configurado corretamente;
    \item O {\it scanner} está ligado e funcionando corretamente;
\end{enumerate}

\paragraph{Pós-condições:} 
\begin{enumerate}
    \item O documento estará digitalizado;
    \item O documento estará indexado para busca;
\end{enumerate}

\paragraph{Cenário de sucesso:}
\begin{enumerate}
    \item O {\it scanner} já foi previamente configurado e está operando corretamente;
    \item O ator colocou um documento no {\it scanner};
    \item O ator ativa o procedimento de digitalização;
    \item O ator muda a página do documento;
    \item O ator realiza os passos 3 e 4 até que todo o documento esteja digitalizado;
    \item O ator decide um nome para o novo documento;
\end{enumerate}

\paragraph{Fluxos alternativos:}
\begin{description}
    \item 1. O {\it scanner} não foi configurado ou não está operando; 
    \item (2-8). O {\it scanner} deixa de operar; 
    \item (1-8). O ator desiste da operação; 
\end{description}

\newpage
\subsubsection{Caso de Uso: Edita Documento}

\paragraph{Descrição:} Nesse caso de uso o ator tem como função selecionar um documento no sistema para alterar suas características.

\paragraph{Pré-condições:}
\begin{enumerate}
    \item Há pelo menos um documento digitalizado;
\end{enumerate}

\paragraph{Pós-condições:} 
\begin{enumerate}
    \item O documento foi alterado;
\end{enumerate}
    
\paragraph{Cenário de sucesso:}
\begin{enumerate}
    \item O ator encontrou o documento;
    \item O ator alterou os dados do documento;
    \item O ator confirmou as alterações;
\end{enumerate}

\paragraph{Fluxos alternativos}
\begin{description}
    \item 1. Não há documentos digitalizados; 
    \item 3. O ator não confirmou as alterações;
\end{description}


\newpage
\subsubsection{Caso de Uso: Remove Documento}

\paragraph{Descrição:} Nesse caso de uso o ator tem como função selecionar um documento para ser removido do sistema.

\paragraph{Pré-condições:}
\begin{enumerate}
    \item Há pelo menos um documento digitalizado;
\end{enumerate}

\paragraph{Pós-condições:} 
\begin{enumerate}
    \item O documento foi removido do sistema;
\end{enumerate}
    
\paragraph{Cenário de sucesso:}
\begin{enumerate}
    \item O ator encontrou o documento;
    \item O ator acionou a remoção do documento;
    \item O ator confirmou a remoção do documento;
\end{enumerate}

\paragraph{Fluxos alternativos}
\begin{description}
    \item 1. Não há documentos digitalizados;
    \item 3. O ator não confirmou a remoção do documento;
\end{description}

\newpage
\subsubsection{Caso de Uso: Adiciona Scanner}
 
\paragraph{Descrição:} Nesse caso de uso o ator tem como função preencher os dados necessários para a adição de um novo {\it scanner} no sistema.

\paragraph{Pré-condições:}
\begin{enumerate}
    \item Há pelo menos um {\it scanner} conectado ao computador onde o sistema está instalado;
\end{enumerate}

\paragraph{Pós-condições:} 
\begin{enumerate}
    \item O novo {\it scanner} está configurado e pronto para uso;
\end{enumerate}

\paragraph{Cenário de sucesso:}
\begin{enumerate}
    \item O ator preencheu os dados do {\it scanner} corretamente;
    \item O sistema encontrou o {\it scanner} que o ator se referiu;
    \item O sistema registrou o novo {\it scanner};
\end{enumerate}

\paragraph{Fluxos alternativos}
\begin{description}
    \item 1. Os dados digitados pelo autor são inválidos; 
    \item 2. Não há {\it scanners} conectados ao computador;
\end{description}

\newpage
\subsubsection{Caso de Uso: Remove Scanner}

\paragraph{Descrição:} Nesse caso de uso o ator tem como função escolher um {\it scanner} para ser removido do sistema.

\paragraph{Pré-condições:}
\begin{enumerate}
    \item Há pelo menos um {\it scanner} registrado no sistema;
\end{enumerate}

\paragraph{Pós-condições:} 
\begin{enumerate}
    \item O {\it scanner} não estará mais registrado no sistema;
\end{enumerate}

\paragraph{Cenário de sucesso:}
\begin{enumerate}
    \item O ator escolheu o  {\it scanner} a ser removido;
    \item O ator confirmou a remoção do {\it scanner} do sistema;
\end{enumerate}

\paragraph{Fluxos alternativos}
\begin{description}
    \item 1. Não há {\it scanners} registrados no computador; 
    \item 2. O ator não confirmou a remoção do {\it scanner};
\end{description}

\newpage
\subsubsection{Caso de Uso: Configurar Scanner}

\paragraph{Descrição:} Nesse caso de uso o ator tem como função escolher um {\it scanner} e em seguida inserir novos dados sobre esse {\it scanner}.

\paragraph{Pré-condições:}
\begin{enumerate}
    \item Há pelo menos um {\it scanner} registrado no sistema;
\end{enumerate}

\paragraph{Pós-condições:} 
\begin{enumerate}
    \item O {\it scanner} estará configurado e pronto para usar;
\end{enumerate}

\paragraph{Cenário de sucesso:}
\begin{enumerate}
    \item O ator escolheu o  {\it scanner} a ser configurado;
    \item O ator configurou o {\it scanner};
    \item O ator confirmou os dados das novas configurações;
\end{enumerate}

\paragraph{Fluxos alternativos}
\begin{description}
    \item 1. Não há {\it scanners} registrados no computador;
    \item 2. As configurações supridas pelo ator não são válidas; 
    \item 3. O ator não confirmou as novas configurações do {\it scanner};
\end{description}




\section{Modelo de dados}
\label{sec:casos_de_uso}

A modelagem de dados foi realizada com o intuito de representar os dados utilizados pelo {\it Webscan}
da maneira mais abstrata possível, sem se preocupar com locais ou métodos de armazenamento.
Para modelar a aplicação foi utilizado o diagrama de entidade-relacionamento (Figura \ref{fig:der}
que faz parte da metodologia de Modelo Entidade-Relacionamento. 

\subsection{Diagrama entidade-relacionamento}
\begin{figure}[ht]
 \centering
  \includegraphics[scale=0.75]{img/der.pdf}
  \caption {Diagrama entidade-relacionamento}
  \label{fig:der}
\end{figure}

\section{Interface}
\label{sec:mockups}

Nesta seção será apresentadas as telas do sistema, tanto para digitalização de novos documentos quanto para a configuração de {\it scanners}. Na seção \ref{sec:mockups_digitalizar}, tem-se os passos para digitalizar um novo documento (storyboard) e as telas que o sistema apresentará para o usuário (mockups). Na seção seguinte (seção \ref{sec:mockups_configurar}), tem-se as telas para configuração de {\it scanners}.

%%%%%%%%%%%%%%%%%%%%%%%%%%%%%%%%%%%%%%%%%%%%%%%%%%%%%%%%%%%%%%%%%%%
\subsection{Digitalizar Documento}
\label{sec:mockups_digitalizar}

Ao iniciar o sistema, o usuário é apresentado com a tela na figura \ref{fig:dig_1}. É interessante destacar o aviso no topo da tela, mostrando que o sistema está procurando por {\it scanners} instalados e configurados no sistema.

\begin{figure}[h]
 \centering
    \setlength\fboxsep{0pt}
    \setlength\fboxrule{0.5pt}
    \fbox{\includegraphics[scale=0.6]{img/mockups/digitalizacao-1.pdf}}
  \caption {Tela inicial do sistema}
  \label{fig:dig_1}
\end{figure}

Caso não haja nenhum {\it scanner} configurado no sistema, um aviso é apresentado ao usuário, pedindo a ele configure o {\it scanner} selecionado para ser usado pelo sistema.

\begin{figure}[h]
 \centering
    \setlength\fboxsep{0pt}
    \setlength\fboxrule{0.5pt}
    \fbox{\includegraphics[scale=0.6]{img/mockups/digitalizacao-2.pdf}}
  \caption {Tela indiciando erro: o {\it scanner} selecionado não está corretamente configurado ou não está ligado}
  \label{fig:dig_2}
\end{figure}

Se o {\it scanner} estiver corretamente configurado e ligado, o usuário pode iniciar a criação de um novo documento, clicando no botão ``Digitalizar página'', apresentado na figura \ref{fig:dig_3}. Nessa situação, o botão ``Gerar documento'' está desativado e emite uma mensagem caso o usuário tente clicá-lo.

\begin{figure}[h]
 \centering
    \setlength\fboxsep{0pt}
    \setlength\fboxrule{0.5pt}
    \fbox{\includegraphics[scale=0.6]{img/mockups/digitalizacao-3.pdf}}
  \caption {Tela indicando o sistema está pronto e o usuário pode começar a digitalizar documentos}
  \label{fig:dig_3}
\end{figure}

Após clicar no botão ``Digitalizar página'', o é apresentada para o usuário uma mensagem para que ele espere a digitalização do documento que está no {\it scanner}, na tela apresentada na figura \ref{fig:dig_4}.

\begin{figure}[h]
 \centering
    \setlength\fboxsep{0pt}
    \setlength\fboxrule{0.5pt}
    \fbox{\includegraphics[scale=0.6]{img/mockups/digitalizacao-4.pdf}}
  \caption {Tela indicando que uma página está sendo digitalizada}
  \label{fig:dig_4}
\end{figure}

Em seguida, na figura \ref{fig:dig_5}, após a digitalização de várias páginas, é exibida pequenas amostras das páginas já digitalizadas e um marcador, indicando se a página deverá ser incluída no novo documento ou não. Após a seleção das páginas, o usuário deve clicar no botão ``Gerar documento''.

\begin{figure}[h]
 \centering
    \setlength\fboxsep{0pt}
    \setlength\fboxrule{0.5pt}
    \fbox{\includegraphics[scale=0.6]{img/mockups/digitalizacao-5.pdf}}
  \caption {Tela mostrando amostras das páginas já digitalizadas}
  \label{fig:dig_5}
\end{figure}

No próximo passo, representado pela figura \ref{fig:dig_6}, o usuário deve escolher então um nome para o documento e uma breve descrição sobre ele. A descrição deste novo documento é opcional.

\begin{figure}[h]
 \centering
    \setlength\fboxsep{0pt}
    \setlength\fboxrule{0.5pt}
    \fbox{\includegraphics[scale=0.4]{img/mockups/digitalizacao-6.pdf}}
  \caption {Tela para a entrada de um nome e descrição para o novo documento}
  \label{fig:dig_6}
\end{figure}

Finalmente, após a criação do documento, o sistema mostra uma confirmação da criação do documento (figura \ref{fig:dig_7}) e indica seu estado. No exemplo, o sistema está pronto para digitalizar um novo documento.

\begin{figure}[h]
 \centering
    \setlength\fboxsep{0pt}
    \setlength\fboxrule{0.5pt}
    \fbox{\includegraphics[scale=0.6]{img/mockups/digitalizacao-7.pdf}}
  \caption {Tela confirmando a criação de um novo documento}
  \label{fig:dig_7}
\end{figure}

A figura \ref{fig:dig_8} mostra a tela no caso em que o usuário digitalizou páginas anteriormente, porém não gerou um documento. Essas páginas ficam armazenadas no sistema e, logo que ele tente digitalizar novos documentos, poderá decidir se quer usar as páginas previamente digitalizadas ou descartá-las, para gerar um novo documento.

\begin{figure}[h]
 \centering
    \setlength\fboxsep{0pt}
    \setlength\fboxrule{0.5pt}
    \fbox{\includegraphics[scale=0.6]{img/mockups/digitalizacao-8.pdf}}
  \caption {Tela mostrando a situação de páginas previamente digitalizadas}
  \label{fig:dig_8}
\end{figure}



%%%%%%%%%%%%%%%%%%%%%%%%%%%%%%%%%%%%%%%%%%%%%%%%%%%%%%%%%%%%%%%%%%%%
\subsection{Configurar Scanner}
\label{sec:mockups_configurar}

Ao clicar no {\it link} ``Configurações'', o ususário deve selecionar qual {\it scanner} ele deseja configurar. Na tela \ref{fig:config_1}, é possível ver uma tela que mostra a escolha de um dispositivo para configuração.

\begin{figure}[h]
 \centering
    \setlength\fboxsep{0pt}
    \setlength\fboxrule{0.5pt}
    \fbox{\includegraphics[scale=0.6]{img/mockups/config-1.pdf}}
  \caption {Tela mostrando a seleção de {\it scanners} para configuração}
  \label{fig:config_1}
\end{figure}

Após a escolha do {\it scanner}, o usuário encontra a tela exibida na figura \ref{fig:config_2}, na qual encontram-se campos para configuração do dispositivo, como tamanho da página, nome e modelo do {\it scanner}. É interessante notar os botões ``+'' e ``-'', para a adição e remoção de {\it scanners}, respectivamente.

\begin{figure}[h]
 \centering
    \setlength\fboxsep{0pt}
    \setlength\fboxrule{0.5pt}
    \fbox{\includegraphics[scale=0.6]{img/mockups/config-2.pdf}}
  \caption {Tela mostrando as configurações de um {\it scanner}}
  \label{fig:config_2}
\end{figure}


\section{Web Services}
\label{sec:web_services}

Nesta seção será descrita toda a {\it API} utilizada para acessar a aplicação
através de {\it Web Services}. A seção \ref{sec:premissas} apresenta as 
premissas que devem ser levadas e a \ref{sec:convencoes} as convenções em conta ao se analisar a {\it API}.  

 
\subsection{Premissas}
\label{sec:premissas}

\begin{enumerate}
    \item Todas os textos entre [] são variáveis;
    \item Todos os retornos são JSON;
    \item O retorno default é um array associativo que contém um código de 
            erro e sua respectiva mensagem. Se o código de erro for 0 (zero)
            quer dizer que não forão encontrados erros;
    \item Os argumentos iniciados com * são opcionais.
\end{enumerate}
 
\subsection{Convenções}
\label{sec:convencoes}

Nesta seção serão utilizados os seguintes conceitos:

\begin{itemize}
    \item Método
    \item Objeto instanciado
    \item Coleção (lista de objetos instanciados)
\end{itemize}

Um texto seguido por / (barra) em uma URL representa membros de um objeto 
instanciado, e os seguidos por . (ponto) representam membros de uma coleção. 
Por exemplo:

O método {\bf modify} é parte de um objeto instanciado do tipo scanner:
\begin{verbatim}
    /scanner/[name]/modify
\end{verbatim}

O método {\bf add} é parde de uma coleção de scanners:
\begin{verbatim}
    /scanner.add
\end{verbatim}

\subsection{Serviços disponíveis}
\label{sec:servicos_disponiveis}

\subsubsection{Scanner}

\begin{itemize}

\item Mostrar os {\it scanners} configurados:
\begin{verbatim}
URL: /scanner
Argumentos: Nenhum
Método: GET, POST
Retorna: Collection containing all scanner objects.
\end{verbatim}

\item Adicionar um novo {\it scanners}:
\begin{verbatim}
URL: /scanner.add
Argumentos: name, localization*, description*, manufacturer, 
    model, papersize*, isindexable*, user*, group*
Método: POST
Retorna: Scanner object.
\end{verbatim}

\item Retornar um objeto {\it scanner}:
\begin{verbatim}
URL: /scanner/[name]
Argumentos: Nenhum
Método: GET, POST
Retorna: Scanner object.
\end{verbatim}

\item Modificar a configuração de um {\it scanner}:
\begin{verbatim}
URL: /scanner/[name]/modify
Argumentos: name*, localization*, description*, manufacturer*, 
    model*, papersize*, isindexable*, user, group
Método: POST
Retorna: Scanner object.
\end{verbatim}

\item Remover um {\it scanner}:
\begin{verbatim}
URL: /scanner/[name]/delete
Argumentos: Nenhum
Método: DELETE
Retorna: Default
\end{verbatim}

\item Retornar o código de status do {\it scanner}:
\begin{verbatim}
URL: /scanner/[name]/status
Argumentos: Nenhum
Método: GET, POST
Retorna: Default
\end{verbatim}

\item Digitalizar uma página:
\begin{verbatim}
URL: /scanner/[name]/scan
Argumentos: Nenhum
Método: GET, POST
Retorna: Page object.
\end{verbatim}

\end{itemize}

\subsubsection{User}

\begin{itemize}

\item Listar todos os usuários registrados:
\begin{verbatim}
URL: /user
Argumentos: Nenhum
Método: GET, POST
Retorna: Collection containing all user objects.
\end{verbatim}

\item Adicionar um novo usuário:
\begin{verbatim}
URL: /user.add
Argumentos: username, password, role
Método: POST
Retorna: User object.
\end{verbatim}

\item Retornar um objeto usuário:
\begin{verbatim}
URL: /user/[username]
Argumentos: Nenhum
Método: GET, POST
Retorna: User object.
\end{verbatim}

\item Modificar o registro de um usuário:
\begin{verbatim}
URL: /user/[username]/modify
Argumentos: username*, password*, role*
Método: POST
Retorna: User object.
\end{verbatim}

\item Deletar um usuário:
\begin{verbatim}
URL: /user/[username]/delete
Argumentos: Nenhum
Método: DELETE
Retorna: Default
\end{verbatim}

\item Listar de todos as páginas digitalizadas e não utilizadas: 
\begin{verbatim}
URL: /user/[username]/page
Argumentos: Nenhum
Método: GET, POST
Retorna: Collection of page objects.
\end{verbatim}

\item Deletar uma página: 
\begin{verbatim}
URL: /user/[username]/page/[pageid]/delete
Argumentos: Nenhum
Método: DELETE
Retorna: Default 
\end{verbatim}

\item Listar os documentos gerados:
\begin{verbatim}
URL: /user/[username]/document
Argumentos: Nenhum
Método: GET, POST
Retorna: Collection of document objects.
\end{verbatim}

\item Deletar um documento: 
\begin{verbatim}
URL: /user/[username]/document/[docname]/delete
Argumentos: Nenhum
Método: DELETE
Retorna: Default 
\end{verbatim}

\item Criar um novo documento usando as páginas selecionadas:
\begin{verbatim}
URL: /user/[username]/document.create
Argumentos: docname, docdescript, pages (lista de pageid)
Método: GET, POST
Retorna: Document object.
\end{verbatim}

\item Baixar um documento digitalizado:
\begin{verbatim}
URL: /user/[username]/document/[docname]
Argumentos: Nenhum
Método: GET, POST
Retorna: Downloadable PDF document.
\end{verbatim}

\end{itemize}

\subsubsection{Auth}

\begin{itemize}

\item Autenticar um usuário:
\begin{verbatim}
URL: /auth/signin
Argumentos: username, password
Método: POST
Retorna: Default
\end{verbatim}

\item Desautenticar um usuário:
\begin{verbatim}
URL: /auth/signout
Argumentos: Nenhum
Método: GET, POST
Retorna: Default
\end{verbatim} 

\end{itemize}

\section{Bibliotecas de Digitalização}
\label{sec:pesquisa_libs}

Uma das atividades na fase atual do projeto foi elaborar uma pesquisa sobre as principais bibliotecas de digitalização de documentos disponíveis para plataformas Microsoft Windows, através da interface TWAIN e para plataforma GNU/Linux, através da interface SANE. 

Na seção \ref{sec:twain}, tem-se as principais bibliotecas encontradas para o uso da interface TWAIN e então, na seção \ref{sec:sane}, tem-se as principais bibliotecas encontradas para o uso da interface SANE.

\subsection{TWAIN}
\label{sec:twain}

\subsubsection{Descrição}
TWAIN é a interface de câmeras digitais e scanners específica para plataforma Windows 32 bits apenas. Não suporta scanners distribuídos na rede e não separa interface do driver (segundo www.sane-project.org).

\subsubsection{Bibliotecas}
\begin{description*}
	\item[Nome:] TWAIN Module
	\item[Linguagem(ns):] Python (versões 2.1 até 2.5)
	\item[Licença:] GPLv2
	\item[Plataforma(s):] Windows 32 bits
	\item[Endereço:] http://twainmodule.sourceforge.net
	\item[Última versão:] 1.0.3
	\item[Data da última atualização do site:] Novembro de 2006
	\item[Data do último {\it release}:] 31 de maio de 2007
	\item[Atividade de desenvolvimento:] parado
	\item[Descrição:] 
	Bem completa. Suporta atividades básicas TWAIN como funções prontas ou funções TWAIN avançadas que podem ser implementadas. Possui documentação ampla, com tutoriais, referências, guias e exemplos.
\end{description*}

\begin{description*}
	\item[Nome:] EZTwain
	\item[Linguagem(ns):] C, C++, Visual Basic, Delphi, entre outras
	\item[Licença:] Domínio público
	\item[Plataforma(s):] Windows 32 bits
	\item[Endereço:] http://www.dosadi.com/eztwain1.htm
	\item[Última versão:] 1.16
	\item[Data da última atualização do site:] Não disponível
	\item[Data do último {\it release}:] 11 de maio de 2007
	\item[Atividade de desenvolvimento:] parado
	\item[Descrição:] 
	É bastante popular, inclusive é amplamente usado através de um {\it wrapper} Java para TWAIN. Documentação limitada, porém possui exemplos prontos em C. Para C, foi testado apenas em Visual C++ 5 e 6.
\end{description*}

\begin{description*}
	\item[Nome:] TWAIN Development Kit
	\item[Linguagem(ns):] C, C++ (Visual Studio)
	\item[Licença:] Privada
	\item[Plataforma(s):] Windows 32 bits
	\item[Endereço:] http://www.twain.org
	\item[Última versão:] Não disponível
	\item[Data da última atualização do site:] Não disponível
	\item[Data do último {\it release}:] Não disponível
	\item[Atividade de desenvolvimento:] Não disponível
	\item[Descrição:] 
	Documentação esparsa, faltam exemplos
\end{description*}

%%%%%%%%%%%%%%%%%%%%%%%%%%%%%%%%%%%%%%%%%%%%%%%%%%%%%%%%%%%%%%%%%%%%%%%%%%%%%%%%%%%%%%%%%%%%%%%%%%%%%%%%%%%%%

\subsection{SANE}
\label{sec:sane}

\subsubsection{Descrição}
SANE (Scanner Access is Now Easy) é uma implementação de aquisição de imagens comum em sistemas open-source, como o Linux e FreeBSD. Há implementações para BeOS, OS/2 e MacOS X.

\subsubsection{Bibliotecas}

\begin{description*}
	\item[Nome:] SANE API
	\item[Linguagem(ns):] C
	\item[Licença:] GPL
	\item[Plataforma(s):] FreeBSD, Linux, BeOS, HP-UX, entre outros.
	\item[Endereço:] http://www.sane-project.org
	\item[Última versão:] 1.0.19
	\item[Data da última atualização do site:] Não disponível
	\item[Data do último {\it release}:] 11 de fevereiro de 2008
	\item[Atividade de desenvolvimento:] alta
	\item[Descrição:] 
	Documentação ampla, comunidade ativa, muitas implementações disponíveis para serem usadas como exemplos.
\end{description*}

\begin{description*}
	\item[Nome:] PIL (Python Imaging Library)
	\item[Linguagem(ns):] Python (versões 2.4 e 2.5)
	\item[Licença:] ver http://www.pythonware.com/products/pil/license.htm
	\item[Plataforma(s):] FreeBSD, Linux, BeOS, HP-UX, entre outros.
	\item[Endereço:] http://www.pythonware.com/products/pil/
	\item[Última versão:] 1.1.6
	\item[Data da última atualização do site:] Não disponível
	\item[Data do último {\it release}:] 3 de dezembro de 2006
	\item[Atividade de desenvolvimento:] Não disponível
	\item[Descrição:] 
	Amplamente usada para processamento de imagens em Linux, usada por diversos front-ends que usam python. Biblioteca simples, manual com tutorial incluso no código fonte da PIL.
\end{description*}



\section{Bibliotecas OCR}
\label{sec:pesquisa_ocr}

\subsection{Introdução}
Uma das atividades do sistema {\it webscan} é fazer o reconhecimento de caracteres usando bibliotecas {\it open-source} de reconhecimento óptico de caracteres, ou mais conhecido pelo termo OCR (Optical Character Recognition). Foram levantadas algumas características das bibliotecas encontradas.

Um fator importante é o reconhecimento de caracteres presentes no português. É interessante que as heurísticas implementadas pelos sistemas pesquisados levem em consideração aspectos como caracteres com acentos agudos, circunflexos, grave ou até mesmo no uso da trema.

\subsection{Informações obtidas}
\label{sec:libs_ocr}

\begin{description*}
    \item[Nome:] GOCR ou JOCR
    \item[Site:] http://jocr.sourceforge.net
    \item[Licença:] GPL
    \item[Descrição:] Tem bom suporte ao idioma inglês, com poucos erros. Não há registro do uso dessa biblioteca com o idioma português.
\end{description*}

\begin{description*}
    \item[Nome:] Conjecture
    \item[Site:] http://www.corollarium.com/conjecture
    \item[Licença:] GPL-2
    \item[Descrição:] {\it Framework} C++ para desenvolvimento de sistemas de OCR, com suporte a módulos genéricos de reconhecimento. A biblioteca GOCR é implementado como módulo para essa {\it framework}. Contém os mesmos problemas da biblioteca GOCR: não há suporte a caracteres acentuados.
\end{description*}

\begin{description*}
    \item[Nome:] Tesseract-OCR
    \item[Site:] http://code.google.com/p/tesseract-ocr/
    \item[Licença:] Apache License 2.0
    \item[Descrição:] Biblioteca antigamente desenvolvida pela HP, atualmente esta sob licença Apache e é conhecida como a melhor biblioteca OCR {\it open-source} da atualidade. Há suporte ao idioma português brasileiro e pode ainda ser treinada para ser melhorada.
\end{description*}




\section{Resultados e Decisões de Projeto}
\label{sec:resultados}

A realização dos trabalhos apresentados neste documento viabiliza a
codificação e implementação do projeto {\emph webscan} de maneira
concisa e objetiva.

Esta seção apresentará as decisões de projeto resultantes dos trabalhos
realizados na etapa 1 do edital \emph{DIGI-DOC}.

\subsection{Decisões de projeto}

\subsubsection{Linguagem de programação}
Para a codificação do {\emph webscan} será utilizada a linguagem Python.
Esta escolha facilitará a integração do sistema desenvolvido com os atuais,
e em desenvolvimento, do Interlegis.

Além disso o Python possibilita um desenvolvimento rápido gerando
um produto estável e com excelente manutenabilidade. 

\subsubsection{Interface}
As interfaces, apresentadas na seção \ref{sec:mockups}, passaram por uma
validação com usuários leigos em computação e com os potenciais usuários
do GITEC apresentando uma taxa de acerto de 87,5\%.

A princípio, a interface seguirá a proposta de telas apresentada,
porém ao longo do desenvolvimento serão realizados novas pesquisas para
tentar identificar pontos onde essas possam ser melhoradas. 

\subsubsection{Web Services}
A especificação dos \emph{web services} apresentada na 
seção \ref{sec:web_services} foi elaborada a partir dos casos de uso
e do modelo de dados também apresentados neste documento.

Essa proposta inicial de métodos e objetos sofrerá adições e subtrações 
ao longo do projeto de acordo com a demanda e amadurecimento do mesmo,
assim como é proposto em um modelo de desenvolvimento iterativo.

\subsubsection{Bibliotecas de digitalização}
Após analisar a pesquisa das bibliotecas de digitalização foi decidido pelo
uso dos módulos \emph{TWAIN Module} e \emph{PIL}.

Os fatores mais relevantes para a escolha de ambas foram o de já possuírem 
implementação em Python e vasta documentação disponível com boa qualidade.

Nesta etapa não foi possível avaliar a estabilidade das bibliotecas, porém
o fato de possuírem licença livre permite que as mesmas possam ser
melhoradas caso o comportamento não seja satisfatório.

\subsubsection{Bibliotecas de OCR}
Para decidir qual biblioteca OCR utilzar só foi necessário utilizar um critério:
suporte da lingua portuguesa e caracteres acentuados.\\ 
A única biblioteca que cumpriu este critério foi a \emph{Tesseract-OCR}.

Apesar de ser a única que atende as necessídades do produto a
\emph{Tesseract-OCR} apresenta diversas outras características úteis, como
suporte a outros 6 idiomas e inteligencia artificial, que possibilita a
melhoria dos resultados ao longo do seu tempo de uso. 


\clearpage

% Glossário
%\section{Glossário}

\subsection{Atividade de Desenvolvimento:}
Atividade de desenvolvimento se refere à quantidade de escritas (ou seja, código sendo atualizado/adicionado) em um sistema de controle de versões, quando disponível.

\begin{description}
	\item[Alta:] diversas atividades no último mês.
	\item[Baixa:] algumas atividades ao longo dos últimos 3 meses.
	\item[Parado:] não houve nenhuma atividade de escrita nos últimos 6 meses.
\end{description}


\end{document}
