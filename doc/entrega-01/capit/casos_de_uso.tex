
\section{Casos de Uso}
\label{sec:casos_de_uso}

Para o projeto, foram elaborados casos de uso do sistema. Na figura \ref{fig:casos_de_uso}
tem-se o diagrama de casos de uso. Na seção \ref{sec:casos_completos} tem-se os casos de uso
completo-abstrato, que indicam as principais atividades que acontecem em cada caso de uso, presentes no diagrama da figura \ref{fig:casos_de_uso}.

\subsection{Diagrama de Casos de Uso}
\begin{figure}[ht]
 \centering
  \includegraphics[scale=0.7]{img/use-case-diagram.pdf}
  \caption {Diagramas de casos de uso}
  \label{fig:casos_de_uso}
\end{figure}

\subsection{Casos de Uso Completos}
\label{sec:casos_completos}

\subsubsection{Caso de Uso: Digitaliza Documento}

\paragraph{Descrição:}
Nesse caso de uso o ator tem como função colocar um documento no {\it scanner} para digitalizá-lo.

\paragraph{Pré-condições:}
\begin{enumerate}
    \item Há um documento no {\it scanner};
    \item O {\it scanner} está configurado corretamente;
    \item O {\it scanner} está ligado e funcionando corretamente;
\end{enumerate}

\paragraph{Pós-condições:} 
\begin{enumerate}
    \item O documento estará digitalizado;
    \item O documento estará indexado para busca;
\end{enumerate}

\paragraph{Cenário de sucesso:}
\begin{enumerate}
    \item O {\it scanner} já foi previamente configurado e está operando corretamente;
    \item O ator colocou um documento no {\it scanner};
    \item O ator ativa o procedimento de digitalização;
    \item O ator muda a página do documento;
    \item O ator realiza os passos 3 e 4 até que todo o documento esteja digitalizado;
    \item O ator decide um nome para o novo documento;
\end{enumerate}

\paragraph{Fluxos alternativos:}
\begin{description}
    \item 1. O {\it scanner} não foi configurado ou não está operando; 
    \item (2-8). O {\it scanner} deixa de operar; 
    \item (1-8). O ator desiste da operação; 
\end{description}

\newpage
\subsubsection{Caso de Uso: Edita Documento}

\paragraph{Descrição:} Nesse caso de uso o ator tem como função selecionar um documento no sistema para alterar suas características.

\paragraph{Pré-condições:}
\begin{enumerate}
    \item Há pelo menos um documento digitalizado;
\end{enumerate}

\paragraph{Pós-condições:} 
\begin{enumerate}
    \item O documento foi alterado;
\end{enumerate}
    
\paragraph{Cenário de sucesso:}
\begin{enumerate}
    \item O ator encontrou o documento;
    \item O ator alterou os dados do documento;
    \item O ator confirmou as alterações;
\end{enumerate}

\paragraph{Fluxos alternativos}
\begin{description}
    \item 1. Não há documentos digitalizados; 
    \item 3. O ator não confirmou as alterações;
\end{description}


\newpage
\subsubsection{Caso de Uso: Remove Documento}

\paragraph{Descrição:} Nesse caso de uso o ator tem como função selecionar um documento para ser removido do sistema.

\paragraph{Pré-condições:}
\begin{enumerate}
    \item Há pelo menos um documento digitalizado;
\end{enumerate}

\paragraph{Pós-condições:} 
\begin{enumerate}
    \item O documento foi removido do sistema;
\end{enumerate}
    
\paragraph{Cenário de sucesso:}
\begin{enumerate}
    \item O ator encontrou o documento;
    \item O ator acionou a remoção do documento;
    \item O ator confirmou a remoção do documento;
\end{enumerate}

\paragraph{Fluxos alternativos}
\begin{description}
    \item 1. Não há documentos digitalizados;
    \item 3. O ator não confirmou a remoção do documento;
\end{description}

\newpage
\subsubsection{Caso de Uso: Adiciona Scanner}
 
\paragraph{Descrição:} Nesse caso de uso o ator tem como função preencher os dados necessários para a adição de um novo {\it scanner} no sistema.

\paragraph{Pré-condições:}
\begin{enumerate}
    \item Há pelo menos um {\it scanner} conectado ao computador onde o sistema está instalado;
\end{enumerate}

\paragraph{Pós-condições:} 
\begin{enumerate}
    \item O novo {\it scanner} está configurado e pronto para uso;
\end{enumerate}

\paragraph{Cenário de sucesso:}
\begin{enumerate}
    \item O ator preencheu os dados do {\it scanner} corretamente;
    \item O sistema encontrou o {\it scanner} que o ator se referiu;
    \item O sistema registrou o novo {\it scanner};
\end{enumerate}

\paragraph{Fluxos alternativos}
\begin{description}
    \item 1. Os dados digitados pelo autor são inválidos; 
    \item 2. Não há {\it scanners} conectados ao computador;
\end{description}

\newpage
\subsubsection{Caso de Uso: Remove Scanner}

\paragraph{Descrição:} Nesse caso de uso o ator tem como função escolher um {\it scanner} para ser removido do sistema.

\paragraph{Pré-condições:}
\begin{enumerate}
    \item Há pelo menos um {\it scanner} registrado no sistema;
\end{enumerate}

\paragraph{Pós-condições:} 
\begin{enumerate}
    \item O {\it scanner} não estará mais registrado no sistema;
\end{enumerate}

\paragraph{Cenário de sucesso:}
\begin{enumerate}
    \item O ator escolheu o  {\it scanner} a ser removido;
    \item O ator confirmou a remoção do {\it scanner} do sistema;
\end{enumerate}

\paragraph{Fluxos alternativos}
\begin{description}
    \item 1. Não há {\it scanners} registrados no computador; 
    \item 2. O ator não confirmou a remoção do {\it scanner};
\end{description}

\newpage
\subsubsection{Caso de Uso: Configurar Scanner}

\paragraph{Descrição:} Nesse caso de uso o ator tem como função escolher um {\it scanner} e em seguida inserir novos dados sobre esse {\it scanner}.

\paragraph{Pré-condições:}
\begin{enumerate}
    \item Há pelo menos um {\it scanner} registrado no sistema;
\end{enumerate}

\paragraph{Pós-condições:} 
\begin{enumerate}
    \item O {\it scanner} estará configurado e pronto para usar;
\end{enumerate}

\paragraph{Cenário de sucesso:}
\begin{enumerate}
    \item O ator escolheu o  {\it scanner} a ser configurado;
    \item O ator configurou o {\it scanner};
    \item O ator confirmou os dados das novas configurações;
\end{enumerate}

\paragraph{Fluxos alternativos}
\begin{description}
    \item 1. Não há {\it scanners} registrados no computador;
    \item 2. As configurações supridas pelo ator não são válidas; 
    \item 3. O ator não confirmou as novas configurações do {\it scanner};
\end{description}


