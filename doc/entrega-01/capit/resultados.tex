\section{Resultados e Decisões de Projeto}
\label{sec:resultados}

A realização dos trabalhos apresentados neste documento viabiliza a
codificação e implementação do projeto {\emph webscan} de maneira
concisa e objetiva.

Esta seção apresentará as decisões de projeto resultantes dos trabalhos
realizados na etapa 1 do edital \emph{DIGI-DOC}.

\subsection{Decisões de projeto}

\subsubsection{Linguagem de programação}
Para a codificação do {\emph webscan} será utilizada a linguagem Python.
Esta escolha facilitará a integração do sistema desenvolvido com os atuais,
e em desenvolvimento, do Interlegis.

Além disso o Python possibilita um desenvolvimento rápido gerando
um produto estável e com excelente manutenabilidade. 

\subsubsection{Interface}
As interfaces, apresentadas na seção \ref{sec:mockups}, passaram por uma
validação com usuários leigos em computação e com os potenciais usuários
do GITEC apresentando uma taxa de acerto de 87,5\%.

A princípio, a interface seguirá a proposta de telas apresentada,
porém ao longo do desenvolvimento serão realizados novas pesquisas para
tentar identificar pontos onde essas possam ser melhoradas. 

\subsubsection{Web Services}
A especificação dos \emph{web services} apresentada na 
seção \ref{sec:web_services} foi elaborada a partir dos casos de uso
e do modelo de dados também apresentados neste documento.

Essa proposta inicial de métodos e objetos sofrerá adições e subtrações 
ao longo do projeto de acordo com a demanda e amadurecimento do mesmo,
assim como é proposto em um modelo de desenvolvimento iterativo.

\subsubsection{Bibliotecas de digitalização}
Após analisar a pesquisa das bibliotecas de digitalização foi decidido pelo
uso dos módulos \emph{TWAIN Module} e \emph{PIL}.

Os fatores mais relevantes para a escolha de ambas foram o de já possuírem 
implementação em Python e vasta documentação disponível com boa qualidade.

Nesta etapa não foi possível avaliar a estabilidade das bibliotecas, porém
o fato de possuírem licença livre permite que as mesmas possam ser
melhoradas caso o comportamento não seja satisfatório.

\subsubsection{Bibliotecas de OCR}
Para decidir qual biblioteca OCR utilzar só foi necessário utilizar um critério:
suporte da lingua portuguesa e caracteres acentuados.\\ 
A única biblioteca que cumpriu este critério foi a \emph{Tesseract-OCR}.

Apesar de ser a única que atende as necessídades do produto a
\emph{Tesseract-OCR} apresenta diversas outras características úteis, como
suporte a outros 6 idiomas e inteligencia artificial, que possibilita a
melhoria dos resultados ao longo do seu tempo de uso. 
